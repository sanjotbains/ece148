\documentclass[12pt]{article}

% Packages
\usepackage{amsmath, amssymb, mathtools}
\usepackage{graphicx}
\usepackage{physics}
\usepackage{geometry}
\usepackage{enumitem}
\usepackage{bm}
\usepackage{listings}
\usepackage{xcolor}
\usepackage{float}

% Geometry settings
\geometry{letterpaper, margin=1in}
\setlength{\parindent}{0pt}

\definecolor{codegreen}{rgb}{0,0.6,0}
\definecolor{codegray}{rgb}{0.5,0.5,0.5}
\definecolor{codepurple}{rgb}{0.58,0,0.82}
\definecolor{backcolour}{rgb}{0.95,0.95,0.92}

\lstdefinestyle{mystyle}{
    backgroundcolor=\color{backcolour},   
    commentstyle=\color{codegreen},
    keywordstyle=\color{magenta},
    numberstyle=\tiny\color{codegray},
    stringstyle=\color{codepurple},
    basicstyle=\ttfamily\footnotesize,
    breakatwhitespace=false,         
    breaklines=true,                 
    captionpos=b,                    
    keepspaces=true,                 
    numbers=left,                    
    numbersep=5pt,                  
    showspaces=false,                
    showstringspaces=false,
    showtabs=false,                  
    tabsize=2
}

\lstset{style=mystyle}

% Title
\title{ECE 148 Homework 6}
\author{Sanjot Bains}
\date{\today}

\begin{document}

\maketitle

\section*{I. Interpolation by DFT}
Consider a real periodic signal $f(t)$ of period $T$. The signal has 7 harmonics, for $m = 1, 2, 3, \ldots, 7$:
\[ f(t) = \sum_{m=1}^7 a_m \sin(m \omega_0 t) \]
where $\omega_0 = \frac{2\pi}{T}$ and the 7 coefficients $\{a_m\}$ are formed with the 7-digit perm number: 9587189.

\begin{lstlisting}[language=Octave, caption=Signal Definition]
T = 1; % Period of the signal
w_0 = 2 * pi / T; % Fundamental frequency
a_m = [9 5 8 7 1 8 9]; % Amplitude of harmonics

f = @(t) a_m(1) * sin(1 * w_0 * t) + a_m(2) * sin(2 * w_0 * t) + ...
         a_m(3) * sin(3 * w_0 * t) + a_m(4) * sin(4 * w_0 * t) + ...
         a_m(5) * sin(5 * w_0 * t) + a_m(6) * sin(6 * w_0 * t) + ...
         a_m(7) * sin(7 * w_0 * t);
\end{lstlisting}

We take 16 uniform samples within one period with sample spacing $\Delta t = \frac{T}{16}$ to form a short 16-point sequence $\{f(n)\}$, where $n = 0, 1, 2, \ldots, 15$.

\begin{lstlisting}[language=Octave, caption=Sampling]
delta_T = T / 16; % Sampling interval
t_samples = 0:delta_T:(T-delta_T); % Sampled time vector
f_n = f(t_samples); % Sampled function values {f(n = 0, 1, ... , 15)}
\end{lstlisting}

Subsequently, we take a 16-point DFT of the sequence to obtain the 16-point spectral sequence F(k), where k = 0, 1, 2, ... , 15.
\[  F(k) = DFT_{N = 16} \{f(n)\}  \]

\begin{lstlisting}[language=Octave, caption=16-pt DFT]
F_k = fft(f_n); % F(k) = DFT of f(n = 0, 1, ... , 15)
F_k_shifted = fftshift(F_k);
\end{lstlisting}


\newpage
\subsection*{1. Interpolation of the DFT Spectrum}
The sequence $f(n)$ is extended to 64 points by padding 48 zeros. The extended sequence $f_a(n)$ is in the form:
\[ f_a(n) = \begin{cases}
f(n) & n = 0, 1, 2, \ldots, 15 \\
0 & n = 16, 17, 18, \ldots, 63
\end{cases} \]

\begin{lstlisting}[language=Octave, caption=Zero-Padded Time Sequence]
f_a = zeros(1, 64);
f_a(1:16) = f_n;
\end{lstlisting}

\vspace{0.5 cm}
We then compute the DFT of the 64-pt sequence:
\[ F_a(k) = DFT_{N = 64} \{f_a(n)\} \]

\begin{lstlisting}[language=Octave, caption=DFT of 64-pt Sequence]
F_a_shifted = fftshift(fft(f_a));
\end{lstlisting}

\begin{figure}[H]
    \centering
    \includegraphics[width=0.8\textwidth]{F_a.png}
    \caption{Comparison of 64-point and 16-point DFT}
\end{figure}


\newpage
\subsection*{2. Interpolation in Time Domain}
The 16-point spectral sequence $F(k)$ is extended to 64 points by inserting 48 zeros in the middle. 

\begin{lstlisting}[language=Octave, caption=Zero-Padded Frequency Sequence]
F_b = zeros(1, 64);
F_b(1:8) = F_k(1:8);
F_b(57:64) = F_k(9:16);
F_b = 4 * F_b;
\end{lstlisting}

\vspace{0.5 cm}
The resulting interpolated time-domain signal $f_b(n)$ is computed using IDFT:
\[  f_b(n) = IDFT_{N = 64} \{F_b(k)\} \]

\begin{lstlisting}[language=Octave, caption=IFFT]
f_b = ifft(F_b);
\end{lstlisting}

\begin{figure}[H]
    \centering
    \includegraphics[width=0.8\textwidth]{f_b.png}
    \caption{Comparison of Interpolated Signal with Original Samples}
\end{figure}


\newpage
\section*{II. Signal Scrambling}
The objective of this exercise is to implement a simple digital speech scrambler.

We use the microphone of of our computer to record a short speech signal $g(t)$, and digitize the speech signal with the A/D tool in Audacity into the discrete form $g(n)$.

\begin{lstlisting}[language=Octave, caption=Reading in Audio File]
[g_n, f_s] = audioread('g_n.wav');
\end{lstlisting}


\subsection*{1. DFT Spectrum}
The DFT spectrum $G(k)$ of the digitized speech signal $g(n)$ is shown below:

\begin{lstlisting}[language=Octave, caption=FFT of Audio Sequence]
G_k = fft(g_n);
G_k_shifted = fftshift(G_k);
\end{lstlisting}

\begin{figure}[H]
    \centering
    \includegraphics[width=0.8\textwidth]{G_k.png}
    \caption{Magnitude Spectrum of Original Speech Signal}
\end{figure}


\newpage
\subsection*{2. Speech Scrambling}
The speech signal is scrambled by multiplying it with an alternating sequence of +1 and -1. 

\begin{lstlisting}[language=Octave, caption=FFT of Hilbert Sequence]
scrambling_seq = ones(size(g_n));
scrambling_seq(2:2:end) = -1;

g_hat_n = g_n .* scrambling_seq;

G_hat_k = fft(g_hat_n);
G_hat_k_shifted = fftshift(G_hat_k);
\end{lstlisting}

\begin{figure}[H]
    \centering
    \includegraphics[width=0.8\textwidth]{G_hat_k.png}
    \caption{Spectrum of Scrambled Speech}
\end{figure}


\newpage
\subsection*{3. Descrambling}
The descrambling process uses the same alternating sequence. When there is a synchronization offset, it results in -g(t) instead of g(t).

\begin{lstlisting}[language=Octave, caption=Descrambling]
descrambled_g_n = g_hat_n .* scrambling_seq;

audiowrite('descrambled_g_n.wav', descrambled_g_n, f_s);
\end{lstlisting}

\newpage
\section*{III. Hilbert Transform}

\subsection*{1. Periodic Signal f(t)}
The Hilbert transform of the periodic signal $f(t)$ is computed and compared with the original signal:

\begin{figure}[H]
    \centering
    \includegraphics[width=0.8\textwidth]{f_t.png}
    \caption{Original Signal and its Hilbert Transform}
\end{figure}


\newpage
\subsection*{2. Speech Signal g(t)}
The Hilbert transform is applied to the speech signal g(t), and the result is saved as an audio file for audible comparison.

\end{document}